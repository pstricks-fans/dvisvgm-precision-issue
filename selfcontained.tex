
% compile it with pdflatex -shell-escape
\documentclass{article}
\usepackage{filecontents}

 %frames.tex
\begin{filecontents*}{frames.tex}
\documentclass[pstricks,border=12pt]{standalone}
\usepackage{pst-plot}
\begin{document}
\multido{\i=10+10}{36}
{
	\begin{pspicture}[showgrid=false](-2,-2)(2,2)
  \psline(2;\i)      
	\end{pspicture}
}
\end{document}
\end{filecontents*}


% animated.tex
\begin{filecontents*}{animated.tex}
\documentclass[border=12pt]{standalone}
\usepackage[dvisvgm]{animate}
\usepackage[dvisvgm]{graphicx}
\begin{document}
\animategraphics[controls,loop,autoplay,scale=1]{10}{frames}{}{}
\end{document}
\end{filecontents*}



\begin{document}


\immediate\write18{latex frames}
\immediate\write18{dvips frames}
\immediate\write18{ps2pdf frames.ps}



% optionally remove temporary files
\makeatletter
\@for\x:={tex,dvi,ps,log,aux}\do{\immediate\write18{cmd /c del frames.\x}}
\makeatother


\immediate\write18{latex animated}


% optionally make sure the existing animated.svg has been deleted.
\immediate\write18{cmd /c del animated.svg}

% there are some peculiarities for --precision=x where x<6.
\immediate\write18{dvisvgm --bbox=papersize --font-format=woff --zoom=-1 --precision=5 animated}

% optionally remove temporary files
\makeatletter
\@for\x:={tex,dvi,ps,log,aux}\do{\immediate\write18{cmd /c del animated.\x}}
\makeatother

Please view the generated animated.svg! 


\end{document}